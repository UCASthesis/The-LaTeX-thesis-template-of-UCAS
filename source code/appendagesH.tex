%空行代表重启一个段落。
%开始插入附录
%\appendix
\chapter{附录H\quad 本研究所用抗体}\label{appen:H}
%直接在奇数页页眉中显示章标题会多处一些章标题内部编号,这里重新定义\leftmark,后续所有章节都要重新定义
\renewcommand{\leftmark}{附录H\quad 本研究所用抗体}
%将章节号计数器设置为8
\setcounter{chapter}{8}
%将插图序号计数器设置为1
\setcounter{figure}{0}
%将表格序号计数器设置为1
\setcounter{table}{0}
%开始表格浮动体环境,其中!表示取消严谨限制,h表示在此处插入,t表示在本页或下一页顶部插入
\begin{table}[!ht]
%调整字号
\small
%居中对齐
\centering
%生成中英双语标题
\bicaption[tab:tableH]{}{本研究所用抗体。}{Table}{Antibodies used in this study.}
%开始绘制表格
\begin{tabular}[c]{*{4}{l} @{}}
%绘制一条水平线
\toprule
%注意这里使用了表格内换行命令,该命令的说明在commands.tex文件中。不推荐用这种方式换行。
\textbf{Immunogen} & \tabincell{l}{\textbf{Biological}\\\textbf{source}} & \tabincell{l}{\textbf{Dilution}\\\textbf{Factor}} & \textbf{Note and Source}\\
\midrule
\tabincell{l}{acetylated\\$\upalpha$-tubulin} & mouse & 1:1000 & Sigma, \#T6793\\
GFP & mouse & 1:1000 & Roche, \#11814460001\\
GFP & mouse & 1:5000 & Abmart, \#M20004\\
HA & rabbit & 1:5000 & Roche, \#11867423001\\
histone H3 & rabbit & 1:5000 & Agrisera, \#AS10710\\
IFT46 & rabbit & 1:5000 & \tabincell{l}{Prepared by Genscript using the N-terminal\\20 amino-acids peptide and purified using the\\same peptide}\\
IFT52 & rabbit & 1:250 & (Deane et al, 2001)\\
IFT74 & rabbit & 1:2500 & (Qin et al, 2004)\\
IFT81 & mouse & 1:250 & (Lucker et al, 2005)\\
mIgG & goat & 1:500 & \tabincell{l}{Alexa Fluor 488 conjugated, Life technologies,\\A-11029}\\
mIgG & goat & 1:5000 & HRP conjugated, Sigma-Aldrich, \#A4416\\
rIgG & goat & 1:500 & \tabincell{l}{Alexa Fluor 594 conjugated, Life technologies,\\\#A-11037}\\
rIgG & goat & 1:5000 & HRP conjugated, Life technologies, \#G21234\\
rIgG & goat & 1:5000 & HRP conjugated, Sigma-Aldrich, \#6154\\
$\upalpha$-tubulin & mouse & 1:200000 & Sigma-Aldrich, \#T9026\\
\bottomrule
%结束绘制表格
\end{tabular}
%结束表格浮动体环境
\end{table}