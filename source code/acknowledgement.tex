%空行代表重启一个段落。
\chapter{致\quad 谢}
%直接在奇数页页眉中显示章标题会多处一些章标题内部编号,这里重新定义\leftmark,后续所有章节都要重新定义
\renewcommand{\leftmark}{致\quad 谢}
本研究得到国家自然科学基金项目\ “Intraflagellar Transport\ 运输纤毛蛋白的分子机理”(项目编号:31371354)的资助,特此致谢。

衷心感谢我的导师黄开耀研究员对我的悉心指导,他广博的科研兴趣、严谨的科学精神和勤奋的工作态度始终是我学习的楷模。黄老师热爱科研,关心学生。他在科学研究和生活上均给予我很大的帮助。再次感谢黄老师,也祝福他的家人健康如意。

感谢德国马普生化研究所的\ Esben Lorentzen\ 研究员和\ Michael Taschner\ 博士,他们的指导和帮助使得本研究更加完善。感谢清华大学潘俊敏教授的帮助,他提供的藻种和技术指导让本研究得以顺利开展。感谢水生生物研究所的王强研究员和上海生命科学研究院植物生理生态研究所的方玉达研究员,他们提供了部分技术指导和仪器。此外,外籍专家\ Kangsup Yoon\ 教授在我攻读博士学位期间也提供了诸多帮助,在此深表谢意。

感谢研究生部和水生生物研究所其他相关老师的帮助,包括但不限于冯玺老师,廖彩萍老师、赵东辉老师,查梅老师、刘伟老师、左艳霞老师、汪艳老师、周芳老师和肖媛老师。诚挚感谢评审委员会和答辩委员会老师的辛勤工作,它们的奉献使得本论文更加完善。

真诚感谢其他老师和同学的指导、讨论和陪伴,他们包括但不限于:邓璇、刘盖、龙欢、熊燕飞、孙慧芳、王坤、曾辉、张晖、程希、尹凤英、柯文婷、徐南南、王亚丽、万磊、缪荣丽、张璠、陈天兵、夏晓玲、王启宇、郑春蕾、赵丽娟、彭钊、边小娟、杨会慧、张宝龙、蒋思琪、段光前、马小翠、冯丽华、张长群、陶彬彬、高合意、黄福青、杨文涛、陈凯、胡金璐、李威、章可可、刘金良、赵永祥、程儒进、刘家亮、程偲、李柱、宋雨林、魏念、刘胜杰、赵巧云、李政政、舒思敏、郑浩、袁莲、程化梅、姚垚和林轶文。

感谢我的家人和朋友。他们的关心帮助和支持让我备受鼓舞。祝福他们健康快乐!

最后,谨以此毕业论文献给吕诗漫小朋友。我有个问题一直想问你,树上的一氧化二氢甜吗?这么爱吃冰糖和“泥巴”会长胖的你造吗?六一儿童节又快到了,今年我打算回去看你跳舞(绝对不去钓鱼)。因为,我毕业了。所以在舞台上要好好表现哦!最后,不要再偷妈妈的口红了。你已经够美了,尽管是想的!
\vspace{3em}

\hspace{25em}吕\quad 波
\vspace{1em}

\hfill 二零一七年五月于武汉东湖之滨

