%空行代表重启一个段落。
\chapter{引\quad 言}
%直接在奇数页页眉中显示章标题会多处一些章标题内部编号,这里重新定义\leftmark,后续所有章节都要重新定义
\renewcommand{\leftmark}{第一章\quad 引\quad 言}
真核细胞利用不同的细胞器执行特定功能,其中之一是纤毛/鞭毛。纤毛是广泛存在于真核细胞表面的毛发状结构,它由基体、轴丝、基质和纤毛膜组成\ \citep{Mizuno2012,Satir2007}。纤毛的功能可分为两大类\ \citep{Satir2007},其一是作为细胞的马达驱动细胞自身或周围的流体/颗粒物运动。另一类是参与细胞对胞内外环境的感知\ \citep{Ishikawa2011,Mourao2016,Singla2006,Wood2015}。 人体纤毛结构或功能的缺陷可导致诸如多囊肾、眼盲、内脏异位及骨骼异常等纤毛病\ \citep{Fliegauf2007,Gerdes2009,Hildebrandt2011,Hildebrandt2007}。

纤毛的形成、维持及信号传导依赖鞭毛内运输(intraflagellar transport, IFT)\citep{Ishikawa2011,Mourao2016,Pedersen2008,Scholey2003}。IFT\ 是轴丝和纤毛膜之间的颗粒物沿纤毛作双向运动\ \citep{Kozminski1993}。正向\ IFT(从基部到顶部)由\ kinesin-2\ 驱动\ \citep{Kozminski1995,Miller2005,Ou2005,Pan2006,Snow2004,Walther1994},反向\ IFT(从顶部到基部)则由胞质动力蛋白\ 2/1b\ 介导\ \citep{Hou2004,Pazour1999}。目前已经鉴定到的\ IFT\ 蛋白有\ 22\ 个,这些蛋白形成了三个独特的复合物,它们分别是\ IFT-A、IFT-B1\ 和\ IFT-B2\ \citep{Taschner2016,Taschner2016a}。\ IFT\ 可以与它携载的货物一起形成周期性的\ IFT\ 火
车\ \citep{Pigino2009,Stepanek2016,Vannuccini2016,Lechtreck2015,Lechtreck2017}。正向\ IFT\ 火车长约\ \SI{233}{\nm},沿二联管中的\ B\ 管运动。反向\ IFT\ 火车长约\
\SI{209}{\nm},沿二联管中的\ A\ 管运动\citep{Pigino2009,Stepanek2016,Vannuccini2016}。此外,IFT\ 蛋白还参与其他生物学过程如囊泡分泌、细胞分裂、免疫突触及微管纳米管的形成等\
\citep{Baldari2010,Borovina2013,Griffiths2010,Inaba2015,Wood2012,Fu2016}。

纤毛内部无蛋白合成系统,大部分纤毛结构蛋白和信号分子必须在细胞体中合成然后通过\ IFT\ 运输到纤毛内部。因此,几乎所有的\ IFT\ 蛋白和马达分子都富集在基体周围。IFT-B\ 蛋白和正向分子马达在基体呈半环三裂弧状,IFT-A\ 蛋白和反向分子马达则呈半环双裂弧分布\ \citep{Brown2015,Deane2001}。而且,IFT-B\ 和\ IFT-A\ 蛋白之间存在部分共定位\ \citep{Brown2015}。然而,作为纤毛形成起始阶段的关键步骤,IFT\ 蛋白基体定位的分子机制尚不明确。由于\ IFT\ 蛋白如\ IFT46\ 在纤毛再生过程中定位在来源于反式高尔基体网络的囊泡上,Woods\ 等人认为\ IFT\ 可能先加载在囊泡上并招募货物然后靶向运输到基体\
\citep{Wood2014}。然而这一观点有待进一步证实。许多纤毛蛋白(尤其是纤毛膜蛋白)含有纤毛定位序列(ciliary targeting sequences, CTSs)\citep{Bhogaraju2013,Malicki2014}。这些\ CTSs\ 介导了目标蛋白的基体和纤毛定位。然而,CTSs\ 种类繁多,如\ G\ 蛋白偶联受体第三个胞内环上的基序、Ax(S/A)xQ\ 基序、RVxP\ 基序、核定位信号(nuclear localization signal, NLS)以及\ SUMOylation。这表明多种不同的运输系统参与了这一过程\ \citep{Bhogaraju2013,Malicki2014,Dishinger2010,McIntyre2015}。尽管如此,IFT\ 蛋白中不存在已知的\ CTSs。

已知某些蛋白如\ BBS7、BBS8、C2cd3、CCDC41、OFD1、Rsg1\ 和\ TTBK2\ 影响特定\ IFT\ 蛋白的基体定位\ \citep{Blacque2004,Brooks2013,Goetz2012,Joo2013,Ye2014}。然而它们的功能不具有特异性\ \citep{Toriyama2016}。2016\ 年,研究人员发现包括\ Jbts17、Intu、Fuz、Wdpcp\ 和\ Rsg1\ 在内的纤毛形成与平面极性效应蛋白(ciliogenesis and planar polarity effectors, CPLANE)可将\ IFT-A\ 的外周亚基招募到基体\ \citep{Brooks2012,Toriyama2016}。尽管\ CPLANE\ 蛋白可能只是影响\ IFT-A\ 外周亚基的组装或稳定性,这仍然是该领域的重大突破之一\ \citep{Toriyama2016}。

本研究中我们关注的是\ IFT46\ 基体定位的分子机制。我们拟通过表达截短片段的方式鉴定\ IFT46\ 的基体和纤毛定位序列,同时我们将通过一系列细胞、生化和遗传学研究与定位序列相互作用的蛋白。最终我们将初步探明\ IFT46\ 基体定位的分子机制。这一研究有助于最终阐明\ IFT\ 蛋白基体定位以及与货物相互作用的机制,同时对调控纤毛形成及纤毛功能也具有重要的指导意义。



