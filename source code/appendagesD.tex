%空行代表重启一个段落。
%开始插入附录
%\appendix
\chapter{附录D\quad 本研究所用溶液}\label{appen:D}
%直接在奇数页页眉中显示章标题会多处一些章标题内部编号,这里重新定义\leftmark,后续所有章节都要重新定义
\renewcommand{\leftmark}{附录D\quad 本研究所用溶液}
%开始列表环境,要更换标志符号请在TeXFriend中选择
\begin{compactitem}[\FourClowerSolid]
\item 1xAnnealing buffer for DNA oligos: 10 mM Tris, 1 mM EDTA, 50 mM NaCl (pH 8.0)\ 或\ 100 mM sodium phosphate, 150 mM NaCl, 1 mM EDTA (pH 7.5)。注意这里是\ 1x\ 的浓度,配置时请根据需要加倍。也可以直接使用\ HiFi(TRANSGEN)中的\ 10xBuffer II。
%ex代表当前字体设置下字母X的高度
\vspace{2ex}
\item 1x\ 电泳缓冲液(1 L):取\ 100 mL 10xTris-甘氨酸缓冲液,加入\ 10 mL 10\%SDS\ 后用去离子水定容至\ 1 L,混匀后室温保存。
\vspace{2ex}
\item 1x\ BSN\ 转膜缓冲液(1 L):取\ \SI{5.82}{\g}\ Tris、\SI{2.93}{\g}\ 甘氨酸和\ \SI{0.375}{\g}\ SDS\ 溶解在\ \SI{500}{\mL}\ 去离子水中,加入\ \SI{200}{\mL}\ 甲醇后定容至\ \SI{1}{\L}。可先配置母液,实际使用时再稀释。
\vspace{2ex}
\item 1xHMDEK:20 mM HEPES,5 mM MgCl$_2$,1 mM EDTA,1 mM DTT。
\vspace{2ex}
\item \SI{100}{\milli\nauticalmile}\ IPTG:取\ \SI{2.4}{\g}\footnote{使用\ GraphPad\ 的\ Molarity Calculator (http://www.graphpad.com/quickcalcs/Molarityform.cfm)\ 计算所需的质量。}\ IPTG\ 溶解在\ \SI{100}{\mL}\ 水中,使用\ \SI{0.22}{\um}\ 的滤膜过滤除菌,分装后\ \SI{-20}{\degreeCelsius}\ 保存备用。
\vspace{2ex}
\item 10\% SDS(W/V)(50 mL):取\ 5 g SDS,加入\ 30 mL\ 去离子水后在微波炉中用小火加热溶解,定容至\ 50 mL\ 后混匀,室温保存。
\vspace{2ex}
\item 1xTAE:取\ 20 mL\ 的\ 50xTAE\ 缓冲液,加去离子水混匀后定容至\ 1 L。
\vspace{2ex}
\item 1xTAPS(0.4\% PEG6000):取\ 0.4 g PEG6000\ 溶解在\ 90 mL\ TAP中,用\ 0.22 $\upmu$m\ 的滤器过滤到无菌的蓝盖瓶中,加入\ 10 mL\ 的\ 10xTAPS\ 后混匀室温保存。
\vspace{2ex}
\item 1xTAPS:取\ 10 mL\ 的\ 10xTAPS\ 加入到\ 90 mL\ TAP中,混匀后室温保存。
\vspace{2ex}
\item 10xTAPS:取\ 10.93 g D-Sorbitol\ 溶解在\ 100 mL\ TAP\ 中,用\ 0.22 $\upmu$m\ 的滤器过滤到无菌的蓝盖瓶中,室温保存。
\vspace{2ex}
\item 10\%\ 过硫酸铵溶液(W/V)(10 mL):取\ 1 g\ 过硫酸铵用少量去离子溶解后定容到\ 10 mL,用\ 1.5 mL\ 离心管\ 500 $\upmu$L\ 分装后于\ \SI{4}{\degreeCelsius}\ 保存。
\vspace{2ex}
\item 10x\ 丽春红染色液:取\ 2 g\ 丽春红\ S、30 g\ 三氯乙酸和\ 30 g\ 磺基水杨酸,加少量去离子水溶解后定容至\ 100 mL,混匀后室温保存。使用前用水稀释十倍。
\vspace{2ex}
\item 10xTBS:取\ 87.6 g NaCl、30 g Tris\ 和\ 2 g KCl,加\ 950 mL\ 去离子水溶解后用浓盐酸调\ pH\ 值为\ 8.0,去离子水定容至\ 1 L,混匀后室温保存。
\vspace{2ex}
\item 10xTris-甘氨酸缓冲液(2 L):取\ Tris 60.58 g,甘氨酸\ 288.4 g,用少量去离子水溶解后定容至\ 2 L,混匀后室温保存。
\vspace{2ex}
\item 2xLaemmli\ 上样缓冲液:取\ 4 mL\ 10\%SDS(W/V),2 mL\ 甘油,1.2 mL 1 M Tris-Cl (pH 6.8)\ 混匀,加入溴酚蓝粉末至其终浓度为\ 0.02\%(W/V),最后用去离子水定容至\ 10 mL。混匀后室温保存。
\vspace{2ex}
\item 20\%\ 淀粉溶液:取\ 4 g\ 淀粉加入到\ 50 mL\ 离心管中,加入\ 20 mL\ 无水乙醇,涡旋混匀后\ 1000 g\ 室温离心一分钟弃上清。用\ 20 mL\ 无菌水洗涤两次,最后用\ 70\%\ 酒精定容到\ 20 mL。
\vspace{2ex}
\item 20\%\ 吐温\ 20:取\ 20 mL\ 吐温\ 20\ 加入\ 80 mL\ 去离子水,混匀后室温保存。
\vspace{2ex}
\item 20 mg/mL X-gal:使用\ DMF\footnote{Dimethylformamide,二甲基甲酰胺}\ 配制\ X-gal,分装后\ \SI{-20}{\degreeCelsius}\ 保存备用。
\vspace{2ex}
\item 30\%\ 凝胶贮液(W/V)(Acr$\cdot$Bis, 291):Bio-Rad\ 原液。该试剂有一定毒性,使用时请注意防护。
\vspace{2ex}
\item 5x样品缓冲液(50 mL):3 mL 1 mol/L Tris-HCl(pH 6.8),25 mL 50\% 甘油,10 mL10\% SDS,2.5 mL $\upbeta$-巯基乙醇,5 mL 1\%\ 溴酚蓝,4.5 mL\ 去离子水。混匀后\ \SI{4}{\degreeCelsius}\ 保存。
\vspace{2ex}
\item 50xTAE:称取\ Tris base 242 g、Na$_2$EDTA$\cdot$2H$_2$O 37.2 g,加入\ 800 mL\ 的去离子水充分搅拌溶解。加入\ 57.1 mL\ 的冰醋酸并充分混匀。加去离子水定容至\ 1 L,室温保存备用。
\vspace{2ex}
\item 氨基黑溶液:取\ 0.5 g\ 氨基黑\ 10B,加入\ 100 mL\ 洗涤缓冲液,混匀后过滤两次,最后用洗涤缓冲液定容到\ 100 mL。
\vspace{2ex}
\item CaCl$_2$$\cdot$甘油溶液:配置\ 0.2 M\ 的\ CaCl$_2$\ 溶液,过滤除菌。将其与\ 20\%\ 的甘油等体积混合。
\vspace{2ex}
\item 分离胶缓冲液(pH 8.8):取\ Tris 181.65 g,加入少量去离子水溶解后再加水至总体积约\ 950 mL,用浓盐酸调\ pH\ 值为\ 8.8\ 后定容至\ 1 L,室温保存。
\vspace{2ex}
\item GST pull-down\ 裂解液:50 mM Phosphate buffer pH 7.5,150 mM NaCl,10\% glycerol
\vspace{2ex}
\item 考马斯亮蓝\ G250\ 染色液(1 L):取\ 100 mg\ 考马斯亮蓝\ G250\ 溶解在少量去离子水中,缓慢加入\ 250 mL\ 异丙醇并搅拌,加入\ 100 mL\ 冰醋酸搅拌混匀后用去离子水定容至\ 1 L。混匀后用普通滤纸过滤室温保存。
\vspace{2ex}
\item Kodak RP X-OMAT\ 显影液:取\ 500 mL\ A\ 液,各加入\ 20 mL\ B\ 液和\ C\ 液,用去离子水定容至\ 1.5 L,混匀后用棕色试剂瓶分装\ \SI{4}{\degreeCelsius}\ 保存。
\vspace{2ex}
\item Millipore\ 化学发光\ HRP\ 底物:使用前取等体积的两种试剂混匀。
\vspace{2ex}
\item MgCl$_2$$\cdot$CaCl$_2$\ 溶液\ (\SI{500}{\mL}):取\ MgCl$_2$$\cdot$6H$_2$O 8.13 g,加入少量去离子水溶解后再加入\ 1.11 g CaCl$_2$,用去离子水定容至\SI{500}{\mL}。MgCl$_2$\ 和\ CaCl$_2$\ 的终浓度分别为\ 80 mM\ 和\ 20 mM。
\vspace{2ex}
\item 浓缩胶缓冲液(pH 6.8):取\ Tris 181.65 g,加入少量去离子水溶解后再加水至总体积约\ 950 mL,用浓盐酸调\ pH\ 值为\ 6.8\ 后定容至\ 1 L,室温保存。
\vspace{2ex}
\item RNaseA (\SI{1}{\ug}/\uL):取\ 35 mg RNaseA\ 粉末加入至含\ 3.2 mL\ 的\ 0.01 M\ 醋酸钠(pH 5.3)的\ 5 mL\ 离心管内。沸水煮十五分钟,室温静置冷却。加入\ 0.1\ 倍体积即\ 0.32 mL\ 的\ Tris-Cl(1 M, pH 8.0)调\ pH\ 至\ 7.4。\SI{-20}{\degreeCelsius}\ 保存备用。
\vspace{2ex}
\item SBA\ 溶液:取\ 1 M DTT\ 和\ 1 M\ 碳酸钠各\ \SI{100}{\uL}\ 至洁净的\ 2 mL\ 离心管内,加入\
\SI{800}{\uL}\ 双蒸水和\
\SI{17}{\uL}\ 的\ 100x\ 蛋白酶抑制剂,混匀后立即使用。
\vspace{2ex}
\item SBB\ 溶液:取\ 5 g SDS\ 粉末,30 g\ 蔗糖,加水溶解并定容至\ 100 mL。如\ SDS\ 不能立即溶解可室温静置过夜再定容。
\vspace{2ex}
\item 山羊血清封闭液:5\% BSA,1\% cold water fish gelatin,10\%\ 山羊血清
\vspace{2ex}
\item TBST\ 溶液:取\ 100 mL 10xTBS,加入\ 2.5 mL 20\%\ 吐温\ 20,用去离子水定容至\ 1 L,混匀后室温保存。
\vspace{2ex}
\item TEMED(四乙基乙二胺):原液。该试剂易燃,有腐蚀性和强神经毒性,操作时请穿实验服并佩戴一次性口罩和手套在通风橱中进行。使用结束后请及时盖紧瓶盖并放回\ \SI{4}{\degreeCelsius}\ 冰箱。
\vspace{2ex}
\item 脱色液(1 L):冰醋酸\ 100 mL,乙醇\ 50 mL,去离子水定容至\ 1 L,混匀后室温保存。
\vspace{2ex}
\item 洗涤缓冲液:取甲醇\ 450 mL,加入\ 50 mL\ 冰醋酸混匀后室温保存。
\vspace{2ex}
\item 医用\ X\ 光胶片定影液:按照产品的操作说明依次将两种粉末溶解在去离子水中并定容至\ 3.5 L,混匀后用棕色试剂瓶分装\ \SI{4}{\degreeCelsius}\ 保存。
\vspace{2ex}
\item 转膜缓冲液:取\ 100 mL 10xTris-甘氨酸缓冲液,加入\ 200 mL\ 无水乙醇,用去离子水定容至\ 1 L,混匀后室温保存。
%结束列表环境
\end{compactitem}