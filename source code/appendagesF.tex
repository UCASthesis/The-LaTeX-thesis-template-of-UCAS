%空行代表重启一个段落。
%开始插入附录
%\appendix
\chapter{附录F\quad 本研究所用质粒}\label{appen:F}
%直接在奇数页页眉中显示章标题会多处一些章标题内部编号,这里重新定义\leftmark,后续所有章节都要重新定义
\renewcommand{\leftmark}{附录F\quad 本研究所用质粒}
%将章节号计数器设置为6
\setcounter{chapter}{6}
%将插图序号计数器设置为1
\setcounter{figure}{0}
%将表格序号计数器设置为1
\setcounter{table}{0}
%开始表格浮动体环境,其中!表示取消严谨限制,h表示在此处插入,t表示在本页或下一页顶部插入
\begin{table}[!ht]
%调整字号
\small
%居中对齐
\centering
%生成中英双语标题
\bicaption[tab:tableH]{表}{本研究所用质粒。}{Table}{Plasmids used in this study.}
%开始绘制表格
\begin{tabular}[c]{*{3}{l} @{}}
%绘制一条水平线
\toprule
\textbf{Plasmid Number} & \textbf{Relevant Genotype or usage}\textbf{$^a$} & \textbf{Source}\\
\midrule
pGEM-T Easy-\textit{IFT46} & pIFT46, \textit{AMP}$^R$ & Dr. Joel Rosenbaum\\
pHK212 & pIFT46::YFP, \textit{AMP}$^R$ & this study\\
pHK214 & pIFT46::YFP, \textit{AMP}$^R$ \textit{PRM}$^R$ & this study\\
pHK231 & pIFT46-N1::YFP, \textit{AMP}$^R$ \textit{PRM}$^R$ & this study\\
pHK232 & pIFT46-N::YFP, \textit{AMP}$^R$ \textit{PRM}$^R$ & this study\\
pHK233 & pIFT46$\Updelta$C1::YFP, \textit{AMP}$^R$ \textit{PRM}$^R$ & this study\\
pHK242 & pIFT46-C1::YFP, \textit{AMP}$^R$ \textit{HYG}$^R$ & this study\\
pHK243 & pIFT46$\Updelta$N1::YFP, \textit{AMP}$^R$ \textit{PRM}$^R$ & this study\\
pHK244 & pIFT46-C::YFP, \textit{AMP}$^R$ \textit{PRM}$^R$ & this study\\
pHK245 & pIFT46-C1::YFP, \textit{AMP}$^R$ \textit{PRM}$^R$ & this study\\
pHK265 & \textit{AMP}$^R$ \textit{HYG}$^R$ & this study\\
pHK266 & pIFT46::YFP, \textit{AMP}$^R$ \textit{HYG}$^R$ & this study\\
pHK267 & pIFT46-C1$^{L285E/L286E}$::YFP, \textit{AMP}$^R$ \textit{HYG}$^R$ & this study\\
pHK268 & pIFT52::YFP, \textit{AMP}$^R$ \textit{HYG}$^R$ & this study\\
pHK281 & pYFP, \textit{AMP}$^R$ \textit{PRM}$^R$ & this study\\
pHK308 & pBBTS1::YFP, \textit{AMP}$^R$ \textit{PRM}$^R$ & this study\\
pHK309 & pBBTS2::YFP, \textit{AMP}$^R$ \textit{PRM}$^R$ & this study\\
pHK310 & pBBTS3::YFP, \textit{AMP}$^R$ \textit{PRM}$^R$ & this study\\
pHK311 & pBBTS4::YFP, \textit{AMP}$^R$ \textit{PRM}$^R$ & this study\\
pHK312 & pBBTS5::YFP, \textit{AMP}$^R$ \textit{PRM}$^R$ & this study\\
pHK313 & pBBTS6::YFP, \textit{AMP}$^R$ \textit{PRM}$^R$ & this study\\
pHK409 & pIFT52::3HA, \textit{AMP}$^R$ \textit{PRM}$^R$ & this study\\
pHK464 & pYFP::NLS, \textit{AMP}$^R$ \textit{PRM}$^R$ & this study\\
pHK473 & pIFT52C::YFP::NLS, \textit{AMP}$^R$ \textit{PRM}$^R$ & this study\\
pHK86 & pGFP-YFP, \textit{AMP}$^R$ \textit{PRM}$^R$ & (Long and Huang, 2012)\\
pHK87 & pGFP-CFP, \textit{AMP}$^R$ \textit{PRM}$^R$ & (Long and Huang, 2012)\\
pHK250 & pIFT52::YFP, \textit{AMP}$^R$ \textit{PRM}$^R$ & this study\\
pHK469 & pYFP::YFP, \textit{AMP}$^R$ \textit{PRM}$^R$ & this study\\
pHK470 & pIFT52::YFP::NLS, \textit{AMP}$^R$ \textit{PRM}$^R$ & this study\\
pHyg3 & \textit{HYG}$^R$ & Dr. Wolfgang Mages\\
pMD18-T & for TA cloning, \textit{AMP}$^R$ & TaKaRa Bio Inc.\\
\bottomrule
%结束绘制表格
\end{tabular}
%结束表格浮动体环境
\end{table}
$^a$\textit{AMP}$^R$, ampicillin resistance;

\makebox[2mm]{}\textit{PRM}$^R$, paromomycin resistance;

\makebox[2mm]{}\textit{HYG}$^R$, hygromycin B resistance 