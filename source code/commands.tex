%设置chapter的目录格式,其余标题使用默认设置
%\titlecontents{chapter}%
%[0em]%
%{\vspace{3mm}\heiti}%
%{\thecontentslabel}%
%{}%
%{\titlerule*[0.2pc]{$\cdot$}\bf\contentspage}

%设置标题的格式,注意left指的是标题内容与左侧版心之间的距离
\dottedcontents{chapter}[4em]{\vspace{3mm}\heiti\bf}{4em}{0.2pc}
\dottedcontents{section}[4em]{\vspace{1.5mm}}{2.2em}{0.2pc}
\dottedcontents{subsection}[7em]{\vspace{0.75mm}}{3em}{0.2pc}
\dottedcontents{subsubsection}[11em]{}{4em}{0.2pc}
%设置chapter标题与上下文之间的间距,更改章标题字号
\CTEXsetup[beforeskip={0pt},afterskip={15pt},nameformat+={\Large},titleformat+={\Large}]{chapter}
%将section设置为左对齐,更改section字号
\CTEXsetup[format+={\flushleft\large},beforeskip={3.25ex plus 1ex minus .2ex},afterskip={1.5ex plus .2ex}]{section}
%更改subsubsection字号
\CTEXsetup[format+={\normalsize},beforeskip={1.5ex plus .2ex minus .2ex}]{subsection}
\CTEXsetup[beforeskip={1.5ex plus .2ex minus .2ex}]{subsubsection}
%设置标题深度为3,这样subsubsection前即可显示编号,否则没有
\setcounter{secnumdepth}{3}
%替换book文类默认的版式为fancy
\pagestyle{fancy}
%清空版式内容以便重新设置
\fancyhf{}
%右页页眉中间为章标题,其中\noupercase命令可避免系统将所有英文转换为大写
\fancyhead[OC]{\color{darkgray} \nouppercase{ \leftmark}}
%左页页眉中间为论文中文标题,其中\color命令可使页眉页脚为灰色(水印效果)
\fancyhead[EC]{\color{darkgray} \nouppercase{衣藻鞭毛内运输蛋白IFT46基体定位机制的研究}}
%右页右侧和左页左侧为页码
\fancyfoot[OR, EL]{\color{darkgray} \thepage}
%重定义\headrule命令使页眉线为灰色
\renewcommand{\headrule}{\color{darkgray} \hrule width\headwidth}
%重定义plain版式使章首页的版式与其他地方一致
\fancypagestyle{plain}{\fancyhf{}%
\fancyhead[OC]{\color{darkgray} \nouppercase{ \leftmark}}%
\fancyhead[EC]{\color{darkgray} \nouppercase{your CHtitle here}}%
\fancyfoot[OR, EL]{\color{darkgray} \thepage}%
\renewcommand{\headrule}{\color{darkgray} \hrule width\headwidth}}
%来自ccaption宏包的命令,将分隔符由:更改为空格
\captiondelim{\space}
%设置三线表中线条上下的空白
\setlength{\abovetopsep}{5pt}
\setlength{\belowrulesep}{0ex}
\setlength{\aboverulesep}{0ex}
\setlength{\belowbottomsep}{0pt}
%行号起始命令
%\linenumbers[1]
%行号字体命令
%\renewcommand{\linenumberfont}{\normalfont \small \sffamily}
%缩短文献条目之间的距离,该命令有natbib宏包提供
%关于中文文献的处理请参考http://blog.sina.com.cn/s/blog_5e16f1770100l3kc.html
\addtolength{\bibsep}{-11pt}
%改善单词、字母的间距。使用之后文档看起来更好,也更容易阅读
%因为这个宏包对单词、字母的调整和字体是有关,所以将其置于命令文档的最后一行
\usepackage{microtype}
%使用该命令用于表格内换行
\newcommand{\tabincell}[2]{\begin{tabular}{@{}#1@{}}#2\end{tabular}}
%创建索引命令
\makeindex
%改变浮动体(这里主要是图和表)标题的字号,注意图片注释的字号也缩小了
\captionnamefont{\small}
\captiontitlefont{\small}
%修改全文行距,其中参数1.2代表单倍行距,1.25代表1.5倍行距,1.667代表双倍行距
%该命令必须在ctex宏包调用命令之后
\linespread{1.667}


